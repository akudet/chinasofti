\documentclass[UTF8, a4paper]{ctexrep}
\usepackage{ctex}
\usepackage{hyperref}
\usepackage{geometry}

\title{电商需求分析说明书}
\date{November 28, 2016}
\author{copied}

\begin{document}
	\maketitle
	
	\chapter{引言}
	\section{系统概述}
	网上购物为商品交易活动提供一个方便的电子平台。该系统分前台会员注册购物部分和后台系统管理部分。销售商通过后台管理系统将商品信息发布在网上,并对整个购物流程进行有效的控制、管理和统计;消费者通过系统前台部分方便快捷的选购需要的商品,享受销售商提供的各种服务。
	
	\chapter{项目概述}
	\section{背景和目标}
	\subsection{开发背景}
	随着互联网经济的到来,电子商务成为一种全新的贸易方式。电子商务渗透到贸易活动的各个阶段,包括信息交换、售前售后服务、销售、电子支付、运输、组建虚拟企业、共享资源等,电子商务的参与者包括消费者、销售商、供货商、企业雇员等等,而电子商务的目的是要实现企业乃至全社会的高效率、低成本的贸易活动。网店就是电子商务的一个典型例子。它为销售商和用户提供一个有效的沟通平台,对销售商来说,他们可以将最新最快最热的商品信息即时反映在网络中,让网民足不出户就可以看到各种各样的商品;而对于消费者来说,网店则为他们提供了方便快捷省时的服务,强大的搜索功能让消费者随心所欲地找到自己需要的商品,轻轻的点击鼠标就可以将喜爱的商品放进购物车,方便的支付方式让消费者在家里就可以享受到送货上门的服务,而会员分级制度使消费者可以买到便宜实惠的商品。
	\subsection{软件定义}
	网上购物管理系统为商品交易活动提供一个方便的电子平台。销售商通过本系统将商品信息发布在网上,并对整个购物流程进行有效的控制、管理和统计,对商店系统进行管理;消费者通过本系统方便快捷的选购需要的商品,享受销售商提供的各种服务。
	\subsection{问题定义}
	该系统针对顾客和管理员分前台和后台,前台主要业务逻辑是实现顾客按不同方式(邮政递送和货到付款)订购商品,选择不同方式获得商品;顾客还能够注册,积分销费,管理自己的购物车和收藏夹,发表商品评论。后台实现商店管理,这些管理职能主要分为:会员管理,订单管理,商品管理,信息反馈管理,消息管理。后台管理员可通过系统中的邮件系统向前台顾客反馈信息,也可以通过获得顾客的注册信息以其他方式与其获得联系。
	
	\section{用户的特点}
	本软件的最终用户将是前台用户(网上购物者),后台管理人员以及系统维护人员。
	\begin{itemize}
		\item 前台用户,只要求有基本的电脑操作知识,互联网知识即可。
		\item 后台管理用户,要求了解基本的电脑操作知识,经过一定时间的使用培训即可。
		\item 系统维护人员,需要熟练掌握SQL SEVER2000管理员操作知识。能够在发生普通的异常情况时,根据使用说明手册进行维护。
	\end{itemize}
	\section{假定和约束}
	\begin{itemize}
		\item 开发经费方面,由于是课程项目,所以无需资金投入,一切都是在项目组成员课外时间完成。
		\item 由于是应用服务程序,一切以用户的需求为最根本的出发点。
		\item 考虑到用户和管理人员的计算机操作水平有限,希望开发出的系统应保证界面友好,操作简单明了,性能可靠,易于维护,可扩展,易于升级。
	\end{itemize}

	\chapter{需求规定}
	\section{前台功能需求}
	前台主要包括,商品展示,购物车管理,我的订单,用户管理,商品管理。
	\subsection{商品展示}
	实现商品预览,商品明细,分类检索功能;具体是(1)该模块主要是显示最新商品,热卖商品,以及打折促销的商品(2)商品分类检索。
	\subsection{购物车}
	实现添加、删除商品,商品数量修改,清空购物车,结算功能。
	\subsection{我的订单}
	实现订单确认,订单列表,删除订单,查询明细功能;具体是
	a.会员可以通过组合搜索或者快速搜索查找所需要的商品,可以查看返回结果中的某一具体商品信息,能够对该商品进行评论,如果暂时不想购买该商品,可以把该商品加入收藏夹,也可以把加入购物车购买该商品,会员可以查看自己的购物车,并对购物车的物品进行修改,生成订单;
	b.生成订单后,可以通过拨打客服热线,取消该订单。
	c.客户确认购买号即生成一个唯一的订单号,客户依此号码可以查询所购商品情况。
	\subsection{用户管理}
	实现登入/登出,用户注册,信息修改功能
	非会员可以通过注册成为网上购物系统会员;会员登录系统后,才能够查看个人信息,才能够对商品进行评论,才能够购买商品;若会员忘记了自己的密码,可以通过注册时候填写的邮箱向系统要回自己的会员密码。对于自己登陆界面风格的管理。
	(2) 会员登录后,可以查看自己账号的相关信息,可以查看以往购买过的商品,感兴趣的新商品,个人信息汇总,修改个人信息,个人消费积分纪录,查看收藏夹,查看个人历史订单等信息。
	\subsection{信息反馈模块}
	该模块将实现对用户留言的管理。 用户留言和评论模块为管理员和用户之间建立起一个信息交流的平台,目的是根据用户的需求,及时得到用户对商品的满意程度。
	
	\section{后台功能需求}
	后台用于管理员对商品的管理,后台提供会员管理模块,订单管理模块,商品管理模块,信息反馈模块,消息发布模块。
	\subsection{用户管理模块}
	该模块将实现会员等级的设置和销费积分与会员等级关系的确定。
	\subsection{订单管理模块}
	该模块将实现订单的查询和订单的处理,生成发货单,并将订单存入数据库以备用户查询和管理员的管理。在库存不足或将取消订单的情况下,管理员与顾客沟通交流,在此模块中将通过一个邮件系统,在特定条件下自动向用户发邮件。
	\subsection{商品管理模块}
	该模块将实现商品入库,商品类型管理,主要完成以下任务:
	添加新的商品,向数据库中添加最新商品,并在首页显示;修改商品,修改商品价格、数量等,以刺激消费者产生消费欲望;删除商品,将一些过期或受召回事件影响的商品下架,以免带来负面影响;查询商品,便于及时掌握商品信息。	
	\subsection{信息反馈模块}
	该模块将实现对用户留言的管理。 用户留言和评论模块为管理员和用户之间建立起一个信息交流的平台,目的是根据用户的需求,及时得到用户对商品的满意程度。
	\subsection{消息发布模块}
	该模块将新商品消息,促销商品,热销商品,本月top 10排行榜发布在前台首页,以供用户快速查找到所需的商品。
	
\end{document}